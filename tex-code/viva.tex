\documentclass[aspectratio=169, hideothersubsections]{beamer}
%\documentclass{beamer}
%\documentclass[aspectratio=1610]{beamer}
\usepackage{listings}
\usepackage{graphicx}
\usepackage{tikz}
\usetheme{Berkeley}
\usefonttheme{structuresmallcapsserif}
\usecolortheme{owl}
%\usepackage{xcolor}
%\usepackage{darkmode}
%\enabledarkmode
%\usecolortheme{albatross}
%\usecolortheme{spruce}
\usepackage{minted}
\usepackage{comment}
\usepackage{animate}  


\setbeamertemplate{page number in head/foot}[totalframenumber]
\setbeamertemplate{navigation symbols}{\footnotesize\usebeamertemplate{page number in head/foot}}
\setbeamercolor{title}{fg = OwlRed}
\setbeamercolor{author}{fg = OwlBlue}
%\setbeamercolor{frametitle}{fg=white}
%\setbeamercolor{background canvas}{bg=black}
%\setbeamercolor{normal text}{fg=white}

%--------------------------------------
%REPLACE IMAGE.JPG WITH LOGO OF ORGANISATION
\logo{\begin{tikzpicture}
\node[inner sep=0pt] at (0,0) {\includegraphics[height=1cm]{SODS Logo.png}};
\end{tikzpicture}}
%--------------------------------------

\title{BSDS-202 : Advanced Python}
\subtitle{Course Instructor - Mr. Nitish Patil}
\author[Gauri Sharan]{Gauri Sharan - BSc Data Science, Semester 4}
\date{June 11, 2024}

\begin{document}
\frame{\titlepage}

\begin{frame}{Table of contents}
    \tableofcontents[hideallsubsections]
\end{frame}

% Unit 1
\section{OOP}

\subsection{Understanding the importance of Object-oriented programming (OOP) concepts}
\begin{frame}
\frametitle{Object-oriented Programming (OOP) Concepts}
OOP is a programming paradigm based on the concept of "objects", which can contain data and code to manipulate that data. Key concepts include:
\begin{itemize}
    \item \textbf{Class}: A blueprint for creating objects.
    \item \textbf{Objects}: Instances of classes.
    \item \textbf{Inheritance}: Mechanism by which one class can inherit attributes and methods from another class.
    \item \textbf{Encapsulation}: Hiding the internal state of an object and requiring all interaction to be performed through an object's methods.
    \item \textbf{Polymorphism}: The ability to present the same interface for different underlying data types.
\end{itemize}
\end{frame}

\begin{frame}[fragile]{Python code example: Class and Objects}
\rule{\textwidth}{1pt}
\scriptsize
\begin{minted}{python}
class Dog:
    def __init__(self, name, age):
        self.name = name
        self.age = age

    def bark(self):
        return f"{self.name} is barking."

my_dog = Dog("Buddy", 3)
print(my_dog.bark())
\end{minted}
\rule{\textwidth}{1pt}
\end{frame}

\subsection{Advanced Python syntax and concepts}
\begin{frame}
\frametitle{Advanced Python Syntax and Concepts}
Advanced features of Python include:
\begin{itemize}
    \item \textbf{Iterators}: Objects that can be iterated upon.
    \item \textbf{Generators}: Functions that return an iterable set of items, one at a time, in a special way.
    \item \textbf{Decorators}: Functions that modify the behavior of another function.
    \item \textbf{Context Managers}: Allow you to allocate and release resources precisely when you want to.
\end{itemize}
\end{frame}

\begin{frame}[fragile]{Python code example: Generators}
\rule{\textwidth}{1pt}
\scriptsize
\begin{minted}{python}
def countdown(n):
    while n > 0:
        yield n
        n -= 1

for count in countdown(5):
    print(count)
\end{minted}
\rule{\textwidth}{1pt}
\end{frame}

\subsection{Functions \& Recursive Function}
\begin{frame}[fragile]
\frametitle{Functions \& Recursive Function}
Functions are blocks of code that only run when called. Recursive functions are functions that call themselves.
\rule{\textwidth}{1pt}
\scriptsize
\begin{minted}{python}
def factorial(n):
    if n == 1:
        return 1
    else:
        return n * factorial(n-1)

print(factorial(5))
\end{minted}
\rule{\textwidth}{1pt}
\end{frame}

% Unit 2
\section{Web Scraping}

\subsection{Understanding the web scraping libraries and functionalities}
\begin{frame}
\frametitle{Web Scraping Libraries and Functionalities}
Web scraping involves fetching and extracting data from websites. Python libraries used for web scraping include:
\begin{itemize}
    \item \textbf{Beautiful Soup}: Parses HTML and XML documents.
    \item \textbf{Requests}: Sends HTTP requests.
    \item \textbf{Scrapy}: An open-source web-crawling framework.
\end{itemize}
\end{frame}

\subsection{Web Scraping with Beautiful Soup}
\begin{frame}[fragile]{Python code example: Web Scraping with Beautiful Soup}
\rule{\textwidth}{1pt}
\scriptsize
\begin{minted}{python}
import requests
from bs4 import BeautifulSoup

url = 'https://example.com'
response = requests.get(url)
soup = BeautifulSoup(response.text, 'html.parser')

for heading in soup.find_all('h2'):
    print(heading.text)
\end{minted}
\rule{\textwidth}{1pt}
\end{frame}

% Unit 3
\section{API Creation}

\subsection{Overview of Python web frameworks such as Flask}
\begin{frame}
\frametitle{Overview of Python Web Frameworks}
Python web frameworks like Flask and Django help in building web applications quickly and efficiently. Flask is a lightweight WSGI web application framework.
\end{frame}

\subsection{Design and implement API using Python web frameworks}
\begin{frame}[fragile]{Python code example: Creating API with Flask}
\rule{\textwidth}{1pt}
\scriptsize
\begin{minted}{python}
from flask import Flask, jsonify

app = Flask(__name__)

@app.route('/api', methods=['GET'])
def api():
    return jsonify({"message": "Hello, World!"})

if __name__ == '__main__':
    app.run(debug=True)
\end{minted}
\rule{\textwidth}{1pt}
\end{frame}

\subsection{Creating RESTful Web APIs using Flask and Python}
\begin{frame}
\frametitle{Creating RESTful Web APIs}
RESTful APIs are based on representational state transfer (REST) technology, an architectural style and approach to communications often used in web services development.
\end{frame}

\subsection{Install Streamlit, Display text with Streamlit}
\begin{frame}[fragile]{Python code example: Display Text with Streamlit}
\rule{\textwidth}{1pt}
\scriptsize
\begin{minted}{python}
import streamlit as st

st.write("Hello, Streamlit!")
\end{minted}
\rule{\textwidth}{1pt}
\end{frame}

\subsection{Display image, audio, video file with Streamlit}
\begin{frame}[fragile]{Python code example: Display Image with Streamlit}
\rule{\textwidth}{1pt}
\scriptsize
\begin{minted}{python}
import streamlit as st

st.image("path/to/image.jpg", caption="Sample Image")
\end{minted}
\rule{\textwidth}{1pt}
\end{frame}

% Unit 4
\section{Tensorflow \& Keras}

\subsection{Tensorflow basics, Tensorflow perceptron, ANN Tensorflow}
\begin{frame}
\frametitle{Tensorflow Basics}
TensorFlow is an open-source framework for machine learning and deep learning. Key concepts include:
\begin{itemize}
    \item \textbf{Tensors}: Multi-dimensional arrays.
    \item \textbf{Graphs}: Define the computation.
    \item \textbf{Sessions}: Execute the graph.
\end{itemize}
\end{frame}

\begin{frame}[fragile]{Python code example: TensorFlow Basics}
\rule{\textwidth}{1pt}
\scriptsize
\begin{minted}{python}
import tensorflow as tf

a = tf.constant(2)
b = tf.constant(3)

with tf.Session() as sess:
    print(sess.run(a + b))
\end{minted}
\rule{\textwidth}{1pt}
\end{frame}

\subsection{Linear Regression}
\begin{frame}[fragile]{Python code example: Linear Regression with TensorFlow}
\rule{\textwidth}{1pt}
\scriptsize
\begin{minted}{python}
import tensorflow as tf

# Define the model
class LinearModel:
    def __init__(self):
        self.W = tf.Variable(tf.random.normal([1]), name='weight')
        self.b = tf.Variable(tf.random.normal([1]), name='bias')

    def __call__(self, x):
        return self.W * x + self.b

# Create the model
model = LinearModel()
# Define a loss function
def loss(predicted_y, target_y):
    return tf.reduce_mean(tf.square(predicted_y - target_y))
#continued in next slide
\end{minted}
\rule{\textwidth}{1pt}
\end{frame}

\begin{frame}[fragile]{Python code example: Linear Regression with TensorFlow}
\rule{\textwidth}{1pt}
\scriptsize
\begin{minted}{python}
# Training data
x_train = [1, 2, 3, 4]
y_train = [0, -1, -2, -3]

# Define a training loop
def train(model, x, y, learning_rate):
    with tf.GradientTape() as t:
        current_loss = loss(model(x), y)
    dW, db = t.gradient(current_loss, [model.W, model.b])
    model.W.assign_sub(learning_rate * dW)
    model.b.assign_sub(learning_rate * db)

# Training the model
epochs = 100
for epoch in range(epochs):
    train(model, x_train, y_train, learning_rate=0.1)
    current_loss = loss(model(x_train), y_train)
    print(f"Epoch {epoch}: Loss: {current_loss.numpy()}")
\end{minted}
\rule{\textwidth}{1pt}
\end{frame}

\subsection{Keras Modules, Layers, Customized Layers}
\begin{frame}[fragile]{Python code example: Keras Layers}
\rule{\textwidth}{1pt}
\scriptsize
\begin{minted}{python}
from tensorflow.keras.models import Sequential
from tensorflow.keras.layers import Dense

model = Sequential([
    Dense(32, activation='relu', input_shape=(784,)),
    Dense(64, activation='relu'),
    Dense(10, activation='softmax')
])

model.compile(optimizer='adam', loss='sparse_categorical_crossentropy', metrics=['accuracy'])
\end{minted}
\rule{\textwidth}{1pt}
\end{frame}

\subsection{Tensorflow \& Keras applications}
\begin{frame}
\frametitle{TensorFlow \& Keras Applications}
Applications of TensorFlow and Keras include image recognition, natural language processing, and predictive analytics.
\end{frame}

\subsection{Projects: MNIST data set, Linear Regression}
\begin{frame}[fragile]{Python code example: MNIST with Keras}
\rule{\textwidth}{1pt}
\scriptsize
\begin{minted}{python}
from tensorflow.keras.datasets import mnist
from tensorflow.keras.models import Sequential
from tensorflow.keras.layers import Dense, Flatten
from tensorflow.keras.utils import to_categorical

# Load dataset
(x_train, y_train), (x_test, y_test) = mnist.load_data()

# Preprocess the data
x_train = x_train.reshape((60000, 28 * 28)).astype('float32') / 255
x_test = x_test.reshape((10000, 28 * 28)).astype('float32') / 255
y_train = to_categorical(y_train)
y_test = to_categorical(y_test)

# Build the model
model = Sequential([
    Flatten(input_shape=(28*28,)),
    Dense(512, activation='relu'),
    Dense(10, activation='softmax')
])
\end{minted}
\rule{\textwidth}{1pt}
\end{frame}

\begin{frame}[fragile]{Python code example: MNIST with Keras}
\rule{\textwidth}{1pt}
\scriptsize
\begin{minted}{python}
# Compile the model
model.compile(optimizer='rmsprop', loss='categorical_crossentropy', metrics=['accuracy'])

# Train the model
model.fit(x_train, y_train, epochs=5, batch_size=128)

# Evaluate the model
test_loss, test_acc = model.evaluate(x_test, y_test)
print('Test accuracy:', test_acc)
\end{minted}
\rule{\textwidth}{1pt}
\end{frame}

% Unit 5
\section{Pytorch}

\subsection{Pytorch Basics, Terminologies, Loading Data}
\begin{frame}
\frametitle{PyTorch Basics}
PyTorch is an open-source machine learning library based on the Torch library. Key concepts include tensors, automatic differentiation, and dynamic computation graphs.
\end{frame}

\subsection{Linear Regression, Implementing Neural Network}
\begin{frame}[fragile]{Python code example: Linear Regression with PyTorch}
\rule{\textwidth}{1pt}
\scriptsize
\begin{minted}{python}
import torch
import torch.nn as nn
import torch.optim as optim

# Data
x_train = torch.tensor([[1.0], [2.0], [3.0], [4.0]])
y_train = torch.tensor([[0.0], [-1.0], [-2.0], [-3.0]])

# Model
class LinearModel(nn.Module):
    def __init__(self):
        super(LinearModel, self).__init__()
        self.linear = nn.Linear(1, 1)

    def forward(self, x):
        return self.linear(x)

model = LinearModel()
\end{minted}
\rule{\textwidth}{1pt}
\end{frame}

\begin{frame}[fragile]{Python code example: Linear Regression with PyTorch}
\rule{\textwidth}{1pt}
\scriptsize
\begin{minted}{python}
# Loss and optimizer
criterion = nn.MSELoss()
optimizer = optim.SGD(model.parameters(), lr=0.01)

# Training loop
for epoch in range(100):
    model.train()
    optimizer.zero_grad()
    outputs = model(x_train)
    loss = criterion(outputs, y_train)
    loss.backward()
    optimizer.step()

    print(f'Epoch [{epoch+1}/100], Loss: {loss.item():.4f}')
\end{minted}
\rule{\textwidth}{1pt}
\end{frame}

\subsection{Pytorch Applications}
\begin{frame}
\frametitle{PyTorch Applications}
PyTorch is used in various applications, including -
\begin{itemize}
    \item Computer vision
    \item Natural language processing
    \item Reinforcement learning
\end{itemize}
\end{frame}

\section{References}
\begin{frame}{References}
  \begin{thebibliography}{10}
    \beamertemplatebookbibitems
    \bibitem{ramalho} Luciano Ramalho. \textit{Fluent Python}. O'Reilly Media, 2015.
    \beamertemplatebookbibitems
    \bibitem{beazley} David Beazley and Brian K. Jones. \textit{Python Cookbook}. O'Reilly Media, 2013.
    \beamertemplatebookbibitems
    \bibitem{mckinney} Wes McKinney. \textit{Python for Data Analysis}. O'Reilly Media, 2017.
    \beamertemplatebookbibitems
    \bibitem{raschka} Sebastian Raschka and Vahid Mirjalili. \textit{Python Machine Learning}. Packt Publishing, 2019.
    \beamertemplatebookbibitems
    \bibitem{python_docs} \textit{Python Documentation}. Available at: \url{https://docs.python.org/3/}
    \beamertemplatebookbibitems
    \bibitem{w3schools} \textit{Python Programming Tutorials}. Available at: \url{https://www.w3schools.com/python/}
    \beamertemplatebookbibitems
    \bibitem{realpython} \textit{Real Python}. Available at: \url{https://realpython.com/}
    \beamertemplatebookbibitems
    \bibitem{pythonweekly} \textit{Python Weekly}. Available at: \url{https://www.pythonweekly.com/}
  \end{thebibliography}
\end{frame}

\section{Thank You}
\begin{frame}{Thank You}
Hope you liked this presentation. \newline \newline
\alert{Gauri Sharan} \newline
Student, School of Data Science \newline
AAFT Noida (Shobhit University) \newline
BSc Data Science 2022-25 \newline
Semester 4, 2024 \newline
\begin{itemize}
    \item LinkedIn: \href{https://www.linkedin.com/in/gauri-sharan}{\bf linkedin.com/in/gauri-sharan} 
    \item GitHub: \href{https://github.com/gaurisharan}{\bf github.com/gaurisharan}
    \item Mail: \href{mailto:gaurisharan123@gmail.com}{\bf gaurisharan123@gmail.com}
\end{itemize}
\end{frame}

\end{document}
